\section{Methodology} 

This Research aims to answer 3 main questions:

\begin{enumerate}
  \item How good are TSFMs on financial time-series data?
  \item In which cases do they perform better/worse?
  \item How to improve TSFMs performance?
\end{enumerate}

Methodology of this research is designed in a way to try best answer these three questions in a scientific, statistically significant manner.

\subsection{Time-series prediction evaluation}

Time-series predictions are evaluated using the traditional regression metrics due to the continuous nature of the target variables. All regression metrics are based on some variation of the prediction error\footnote{Prediction error is the difference between the value we predicted and the actual value}. From the prediction error, following metrics are derived, and commonly used in time-series prediction tasks in Finance\cite{hu2021survey}:

\begin{itemize}
  \item RMSE (Root Mean Square Error)
  \item MAE (Mean Absolute Error)
  \item MAPE (Mean Absolute Percentage Error)
  \item R2
\end{itemize}

An additional metric that will be used in this research is MDA (Mean Directional Accuracy)\footnote{MDA is time-series specific. Since time-series data is inherently ordered (as opposed to regular regression data), we can measure the direction in which a time-series is moving at each time-step. Hence we can compare the direction of movement of our predicted time-series against the actual time-series and we can calculate in what percentage of cases did the two time-series move in the same direction (up or down).}. See \cite{costantini2016forecasting} for Directional accuracy calculation.

This is especially important in the field of Finance as in many cases (such as with stock prices) we are not necessarily primarily interested in predicting the actual values, but rather we are interested in predicting the direction in which the price will go next time-step.

\subsection{Data}

This project uses 4 types of financial time-series data:
\begin{enumerate}
  \item Stock index prices
  \begin{enumerate}
      \item S\&P 500
      \item FTSE 100
      \item DOW Jones
      \item NASDAQ
  \end{enumerate}
  \item Exchange rates
  \begin{enumerate}
      \item USD/GBP
  \end{enumerate}
  \item Commodity prices
  \begin{enumerate}
      \item WTI (Oil)
  \end{enumerate}
\end{enumerate}

The specific time-series were chosen solely based on their prominence in the field of Finance.

\subsubsection{Data Pre-processing}

In the field of Finance, there is a concept of "return". Return of an asset A at a certain point in time T: A\textsubscript{T} is calculated as \[(A_T-A_{T-1})/A_{T-1}\] i.e. the percentage change in the value of that asset between the points in time T and T-1. If we think about each one of the fore-mentioned time-series as assets, we can represent them as time-series of returns rather than their actual values which is a common practice in financial time-series prediction \cite{hu2021survey}.


\subsubsection{Time-series data characteristics}




Time-series data exhibits several key characteristics that are essential for effective analysis and forecasting.

\begin{itemize}
    \item Trend\footnote{Trend is a long-term increase or decrease in the data. It could take many shapes: linear, exponential, polynomial, damped.}
    \item Seasonality\footnote{Seasonality is the presence of recurring patterns at regular intervals in the data.}
    \item Noise\footnote{Noise or randomness is short-term fluctuations in the data that do not follow any discernible pattern.}
    \item Stationarity\footnote{A time-series is said to be stationary if the statistical properties such as mean and variance remain constant over time.}
    \item Volatility\footnote{Volatility is degree of variation within the time series.}
    \item Autocorrelation\footnote{Autocorrelation is the degree to which previous values affect future values in a time-series.}
\end{itemize}



Depending on the type and frequency of a financial time-series, it may exhibit different characteristics, therefore, it is important to run experiments across different types of data and different frequencies to find out how well the models handle different time-series characteristics.



\subsection{Experiment Design}

\subsubsection{Benchmark}

Calculating the evaluation metrics for time-series prediction is not sufficient to answer whether a time-series prediction model is good or not. The results of the evaluation need to be put into context and compared against a benchmark in order for us to be able to say whether the results are good or not. 

Chosen benchmark models are: autoARIMA \cite{hyndman2008automatic}\footnote{Originally, this package is for R. I used the equivalent python implementation: https://github.com/alkaline-ml/pmdarima.}, simple autoregressor and Meta's Prophet model \cite{taylor2018forecasting}.

\subsubsection{Time Series Cross Validation}

K-fold Time Series Cross Validation (TSCV) is a statistical method for ensuring statistical significance of a time-series prediction task. In this experiment, I'm using the rolling window version of the TSCV method. Let's say we have chosen the length of the training set to be L, the prediction horizon (how many steps into the future we're predicting) to be H and the time-series of interest has the length L. On the first fold of the TSCV process, we will choose the points 1, 2, ..., L from the time-series to be the "train" set (window) and the points L+1, L+2, ..., L+H to be the "test" set (window) against which we evaluate our models' predictions. (In the field of finance, we are usually only interested in predicting just the next value in the future therefore, in this research, I will use just H=1.) On the next fold of the TSCV, we move the train and test window by one point so we choose points 2, 3, ..., L+1 to be the "train" set/window and the points L+2, L+3, ..., L+H+1 to be the "test" set/window. This procedure is repeated K times. By repeating the prediction and evaluation process many times, we hope that the aggregated evaluation of the model's performance that we calculate is statistically representative of that model's actual performance.




